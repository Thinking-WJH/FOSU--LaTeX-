\include{settings}
% 引入设置文档settings.tex(不需修改)
% 右上方展示你的照片:删除images文件夹中的avatar图片,上传你自己的照片改为avatar.png,其余的别改动。
\newcommand{\school}{物理与光电工程学院}
% 改为自己的学院,学院英文名不长的可在中文名后加上,用“|”隔开。
\newcommand{\contact}{
    \scriptsize
    \textcolor{white}{
        \faEnvelope \quad \href{mailto:youremail@fosu.edu.com}{xxxx@xxxmail.com}
        \hspace{4em}
        \faPhone \quad xxx-xxxx-xxxx
        \hspace{4em}
        \faGithub \quad \href{https://github.com/Thinking-WJH/FOSU-CV}{GitHub 项目地址}
    }
}

\begin{document}

% 页眉设计
\begin{tikzpicture}[remember picture, overlay]
    \node[anchor=north, inner sep=0pt](header) at (current page.north){
        \includegraphics[width=\paperwidth]{images/header.png}
    };
    \node[anchor=west](school_logo) at (header.west){\hspace{0.5cm}};
    \node[anchor=east](school_name) at(header.east){
        \textcolor{white}{\textbf{\school}}\hspace{0.5cm}
    };
\end{tikzpicture}
\vspace{-3.5em}

% 页脚设计
\begin{tikzpicture}[remember picture, overlay]
    % 页脚图片节点
    \node[anchor=south, inner sep=0pt](footer) at (current page.south){
        \includegraphics[width=\paperwidth]{images/footer.png}
    };
    % 联系方式节点
    \node[anchor=center] at (footer.center) {\contact};
\end{tikzpicture}

% 页面背景水印,佛大官方logo,不需改动,不想要水印整模块删除即可
\begin{tikzpicture}[remember picture, overlay]
    \node[opacity=0.05] at(current page.center){
        \includegraphics[width=0.7\paperwidth, keepaspectratio]{images/fosu_logo_big.png}
    };
\end{tikzpicture}

% 个人信息部分
\begin{figure}[h]
    \begin{minipage}{0.82\textwidth}
        \section{\makebox[\widthof{\faAddressCard}][c]{\color{FOSU_Red}{\faAddressCard}}\quad 个人信息}
        \begin{tabularx}{\linewidth}{p{\widthof{出生日期:}}Xp{\widthof{政治面貌:}}X}
            姓\ \ \ \ \ \ \ \ 名: & 你的名字 & 
            性\ \ \ \ \ \ \ \ 别: & 你的性别  \\
            出生年月:              & 年月 & 
            政治面貌:              & 预备党员 \\
           
        \end{tabularx}
    \end{minipage}
    \hspace{2em}
    \begin{minipage}{0.12\textwidth}   
    % 头像照片不需一定正方形,长方形也是自动插入。
    % 调整0.12可调整图片宽度,但需同步调整上文中的\begin{minipage}{0.82\textwidth}中的0.82需要一增一减
        \setlength{\fboxsep}{0pt}
        \doublebox{\includegraphics[width=\linewidth]{images/avatar.png}}
    \end{minipage}

\end{figure}
\vspace{-1em}

% 教育背景部分
\section{\makebox[\widthof{\faGraduationCap}][c]{\color{FOSU_Red}{\faGraduationCap}}\quad 教育背景}
\vspace{0.5em}
{\large \textbf{佛山大学}},本科 \hfill {广东,佛山} \\
{{\href{https://www.fosu.edu.cn/spoe/}{物理与光电工程学院}}},专业:光电信息科学与工程 \hfill {2024年9月-2024年9月} \\
\textbf{主修课程}:物理光学、几何光学、光纤通信、信息光学\ 等。


\vspace{0.5em}
{\large \textbf{佛山大学}},硕士 \hfill {广东,佛山} \\
{{\href{https://www.fosu.edu.cn/spoe/}{物理与光电工程学院}}},专业:光学工程 \hfill {2024年9月-2024年9月} \\
\textbf{研究方向}:1、2\ 等。

\vspace{0.5em}
{\large \textbf{佛山大学}},博士 \hfill {广东,佛山} \\
{{\href{https://www.fosu.edu.cn/spoe/}{物理与光电工程学院}}},专业:光学工程 \hfill {2024年9月-2024年9月} \\
\textbf{研究方向}:1、2\ 等。

% 科研成果部分

\section{\makebox[\widthof{\faGraduationCap}][c]{\color{FOSU_Red}{\faGraduationCap}}\quad 科研成果}
\vspace{0.5em}
This is One of Your Paper Published in Conference A. \\
\textbf{Mingzi Nide}, Daoshi Nide. \hfill 发表于 \textbf{Conference A}(CCF-A类会议)

\vspace{0.5em}
This is Another Paper. \\
\textbf{Mingzi Nide}, Shidi Nide, Daoshi Nide. \hfill 发表于 \textbf{Conference B} (CCF-A类会议)

% 项目与教学部分

\section{\makebox[\widthof{\faChalkboardTeacher}][c]{\color{FOSU_Red}{\faChalkboardTeacher}}\quad 项目与教学}
\vspace{0.5em}
{\large{\textbf{项目名称}}} \hfill {横向/纵向项目-已完结/进行中}\\
\textbf{你在项目中扮演的角色} \hfill 2020年9月至2021年9月\\
项目简介。

\vspace{1em}
{\large{\textbf{某主题竞赛}}},主讲/参与 \hfill {2020年夏季}\\
主要内容:内容1,内容2,内容3\ 等。

\vspace{1em}
{\large{\textbf{课程名称}}},助教 \hfill {2021年夏季}\\
主要内容:内容1,内容2,内容3\ 等。

% 技能部分

% 如果每行的内容不是很多,可以考虑使用 minipage,将内容分列展示
\begin{minipage}[t]{0.6\textwidth}
    \section[技能特长]{\makebox[\widthof{\faWrench}][c]{\color{FOSU_Red}{\faWrench}}\quad 技能特长}
    \begin{itemize}
    \setlength{\itemsep}{0.5em}
        \item 英语:取得CET‑4与CET‑6证书;
        \item 熟练使用 Tensorflow、Pytorch 等深度学习框架。
        \item 软件:能熟练运用CAD,Python,MATLAB,C++等软件
    \end{itemize}
\end{minipage}
\hfill
\begin{minipage}[t]{0.35\textwidth}
    \section[兴趣爱好]{\makebox[\widthof{\faWrench}][c]{\color{FOSU_Red}{\faStar}}\quad 兴趣爱好}
    \begin{itemize}
    \setlength{\itemsep}{0.5em}
        \item 唱
        \item 跳
        \item Rap
        \item 篮球
    \end{itemize}
\end{minipage}
  
\end{document}

